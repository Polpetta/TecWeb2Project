\subsection{Home page}
Come già visto nell'analisi generale del sito, l'\textit{home page} presenta
una grande quantità di contenuto informativo, e qui di seguito si proseguirà con
l'analisi.

Partendo a fare un'analisi dall'header verso il footer, è possibile notare
subito come sia stata privilegiata l'immagine rispetto al contenuto testuale:
è presente infatti uno slider che ricopre in maniera principale la prima
schermata. È importante far notare come in un sito web, rispetto
per esempio ad un giornale, risulti più importante il testo rispetto alle
immagini.

Dopo lo slider è presente una sezione di ``Ultime News'', dove si trovano
una serie di riquadri composti da una piccola immagine di anteprima con il
titolo dell'articolo e l'ora di pubblicazione. Questa sezione a mio avviso va
bene. Le immagini in questo caso fungono da ``calamite'' per i click,
invogliando l'utente a premerci su di esse.

A metà pagina è presente un'inserzione pubblicitaria. Questa inserzione
pubblicitaria però è poco integrata nel sito web ed è facilmente riconoscibile.
Questo causa da parte degli occhi dell'utente un ``filtro'' automatico,
rendendola di fatto poco utile.

Nella sezione ``Dichiarazioni dei piloti'' viene messo in primo piano un
determinato articolo, mentre agli altri seguono in una disposizione a griglia
con immagini di anteprima rimpicciolita, che potrebbe porre delle difficoltà
all'utente, in quanto risultano molto minute.

A seguito è presente una sezione apposita riguardo al gran premio appena
trascorso e ``Analisi della redazione'', che segue lo stesso layout della parte
``Ultime News''.

Infine si trovano due parti disposte verticalmente, rispettivamente per la
``Storia dei gran premi'' e per i ``Video''. Queste parti disposte in maniera
verticale nella pagina a mio avviso la allungano, causando un aumento degli
scroll e quindi una maggiore probabilità che un utente medio non vada a visitare
questa porzione di pagina. È noto infatti che durante la navigazione lo scroll
medio risulta essere 1,3 schermi in più rispetto a quello che si sta già
vedendo. Le probabilità di scroll per chi visita una homepage per la prima
volta sono pari al 23\%, mentre per un utenza abituale la percentuale di scroll
si abbassa al 14\%.

Per rendere più fruibile l'home page a mio avviso sarebbe meglio snellirla,
rimuovendo alcuni contenuti per ridurre il numero di scroll necessari.
