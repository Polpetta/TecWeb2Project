\section{Aspetti genrali}

\subsection{Analisi del nome}
Dal punto di vista sociale \`e fondamentale scegliere un buon nome per un indirizzo web. Ci sono alcune regole che massimizzano il potenziale successo di un sito. In media il 10-20\%, ma \`e possibile arrivare fino ad un +40-50\% di influenza.

\textit{f1world.it} presenta delle buone caratteristiche a livello di nome, infatti:
\begin{itemize}

\item \`e un nome corto, quindi risulta di facile memorizzazione per la maggior parte dell'utenza;
\item il nome risulta relativamente facile da memorizzare, in quanto composto da due parole (e facilmente memorizzabili anche per chi non parla inglese), \textit{f1} e \textit{world};
\item l'iniziale, essendo una \textit{f}, conferisce un bonus del +3.3\%;
\item all'interno dell'indirizzo \`e presente un numero, che conferisce un impatto bonus del +8.2\%.
\end{itemize}

A mio avviso il nome potrebbe aver avuto un impatto maggiore se avesse utilizzato il dominio \textit{.com}.


\subsection{Elementi generali del sito web}

Di seguito verranno analizzati gli elementi in comune di tutto il sito web. Viene presa come riferimento la homepage per questa analisi, ma pi\`u avanti si far\`a un'analisi focalizzata solamente su di essa.

\begin{figure}[H]
  \centering
  \includegraphics[height=18cm, width=10cm]{res/img/HomePage_Full}
  \caption{Homepage del sito, dove verranno analizzati solamente gli elementi che comuni a tutte le pagine.}
\end{figure}

L'analisi partir\`a dalla parte alta del sito per poi scendere fino in fondo.

\subsubsection{Header}

Come si pu\`o notare, all'inizio della pagina \`e presente un sottile header. Nella sinistra dell'header troviamo dei link che ci portano a sezioni secondarie rispetto ai temi trattati nel sito, mentre a destra troviamo delle icone per la condivisione del sito web nei \textit{social network} e un box per la ricerca.
A mio avviso, questa disposizione degli elementi pu\`o essere migliorata. I link presenti nella parte sinistra potrebbero essere spostati nella parte bassa del sito web, per dare possibilit\`a alle icone di condivisione di essere spostate sulla sinistra, lasciando quindi pi\`u spazio alla barra di ricerca, che in questo modo diventerebbe pi\`u visibile all'utente e permetterebbe un migliore inserimento della \textit{query} di ricerca. \`E infatti importante ricordare come per una barra di ricerca sia importate lo spazio e la posizione: la lunghezza media consigliata \`e di almeno 30 caratteri. Barre di ricerca troppo piccole causano stress all'utente, che tendono a inserire \textit{query} di ricerca pi\`u brevi diminuendo l'efficacia del motore di ricerca.


Subito sotto il piccolo header si trova il logo del sito web: questo logo \`e composto da testo, il che aiuta l'utente a diminuire il senso di disorientamento rispetto a un logo senza testo.
Accanto al logo \`e possibile notare uno spazio bianco, che a mio avviso potrebbe essere riempito con del contenuto informativo.


Dopo il logo troviamo un men\`u orizzontale. Questo men\`u a tendina risulta comodo per navigare nelle sezioni del sito: le voci si espandono al passaggio del mouse (\textit{hover}), permettendo all'utente di eplorare comodamente i sotto-argomenti trattati. Non \`e stato tenuto conto dell'algoritmo di \textit{path finding} usato dall'utente per selezionare una delle varie voci: al primo errore o alla prima fuoriuscita accidentale dal men\`u questo si chiude subito, causando una forte irritazione da parte dell'utente che si ritrova costretto a ripercorrere tutta la path svolta fino alla voce desiderata (si ricorda che statisticamente il 92\% delle volte si esce dal cammino del men\`u). Sarebbe meglio introdurre un men\`u \textit{fault-tolerant} che permetta all'utente di non forzare il suo algoritmo di \textit{path finding}. Buono il numero di livelli del men\`u che si ferma a due, rendendo meno impegnativa l'esplorazione delle sotto-categorie.

% TODO:
% - 3 columns layout
% - footer
